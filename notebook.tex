
% Default to the notebook output style

    


% Inherit from the specified cell style.




    
\documentclass[11pt]{article}

    
    
    \usepackage[T1]{fontenc}
    % Nicer default font (+ math font) than Computer Modern for most use cases
    \usepackage{mathpazo}

    % Basic figure setup, for now with no caption control since it's done
    % automatically by Pandoc (which extracts ![](path) syntax from Markdown).
    \usepackage{graphicx}
    % We will generate all images so they have a width \maxwidth. This means
    % that they will get their normal width if they fit onto the page, but
    % are scaled down if they would overflow the margins.
    \makeatletter
    \def\maxwidth{\ifdim\Gin@nat@width>\linewidth\linewidth
    \else\Gin@nat@width\fi}
    \makeatother
    \let\Oldincludegraphics\includegraphics
    % Set max figure width to be 80% of text width, for now hardcoded.
    \renewcommand{\includegraphics}[1]{\Oldincludegraphics[width=.8\maxwidth]{#1}}
    % Ensure that by default, figures have no caption (until we provide a
    % proper Figure object with a Caption API and a way to capture that
    % in the conversion process - todo).
    \usepackage{caption}
    \DeclareCaptionLabelFormat{nolabel}{}
    \captionsetup{labelformat=nolabel}

    \usepackage{adjustbox} % Used to constrain images to a maximum size 
    \usepackage{xcolor} % Allow colors to be defined
    \usepackage{enumerate} % Needed for markdown enumerations to work
    \usepackage{geometry} % Used to adjust the document margins
    \usepackage{amsmath} % Equations
    \usepackage{amssymb} % Equations
    \usepackage{textcomp} % defines textquotesingle
    % Hack from http://tex.stackexchange.com/a/47451/13684:
    \AtBeginDocument{%
        \def\PYZsq{\textquotesingle}% Upright quotes in Pygmentized code
    }
    \usepackage{upquote} % Upright quotes for verbatim code
    \usepackage{eurosym} % defines \euro
    \usepackage[mathletters]{ucs} % Extended unicode (utf-8) support
    \usepackage[utf8x]{inputenc} % Allow utf-8 characters in the tex document
    \usepackage{fancyvrb} % verbatim replacement that allows latex
    \usepackage{grffile} % extends the file name processing of package graphics 
                         % to support a larger range 
    % The hyperref package gives us a pdf with properly built
    % internal navigation ('pdf bookmarks' for the table of contents,
    % internal cross-reference links, web links for URLs, etc.)
    \usepackage{hyperref}
    \usepackage{longtable} % longtable support required by pandoc >1.10
    \usepackage{booktabs}  % table support for pandoc > 1.12.2
    \usepackage[inline]{enumitem} % IRkernel/repr support (it uses the enumerate* environment)
    \usepackage[normalem]{ulem} % ulem is needed to support strikethroughs (\sout)
                                % normalem makes italics be italics, not underlines
    

    
    
    % Colors for the hyperref package
    \definecolor{urlcolor}{rgb}{0,.145,.698}
    \definecolor{linkcolor}{rgb}{.71,0.21,0.01}
    \definecolor{citecolor}{rgb}{.12,.54,.11}

    % ANSI colors
    \definecolor{ansi-black}{HTML}{3E424D}
    \definecolor{ansi-black-intense}{HTML}{282C36}
    \definecolor{ansi-red}{HTML}{E75C58}
    \definecolor{ansi-red-intense}{HTML}{B22B31}
    \definecolor{ansi-green}{HTML}{00A250}
    \definecolor{ansi-green-intense}{HTML}{007427}
    \definecolor{ansi-yellow}{HTML}{DDB62B}
    \definecolor{ansi-yellow-intense}{HTML}{B27D12}
    \definecolor{ansi-blue}{HTML}{208FFB}
    \definecolor{ansi-blue-intense}{HTML}{0065CA}
    \definecolor{ansi-magenta}{HTML}{D160C4}
    \definecolor{ansi-magenta-intense}{HTML}{A03196}
    \definecolor{ansi-cyan}{HTML}{60C6C8}
    \definecolor{ansi-cyan-intense}{HTML}{258F8F}
    \definecolor{ansi-white}{HTML}{C5C1B4}
    \definecolor{ansi-white-intense}{HTML}{A1A6B2}

    % commands and environments needed by pandoc snippets
    % extracted from the output of `pandoc -s`
    \providecommand{\tightlist}{%
      \setlength{\itemsep}{0pt}\setlength{\parskip}{0pt}}
    \DefineVerbatimEnvironment{Highlighting}{Verbatim}{commandchars=\\\{\}}
    % Add ',fontsize=\small' for more characters per line
    \newenvironment{Shaded}{}{}
    \newcommand{\KeywordTok}[1]{\textcolor[rgb]{0.00,0.44,0.13}{\textbf{{#1}}}}
    \newcommand{\DataTypeTok}[1]{\textcolor[rgb]{0.56,0.13,0.00}{{#1}}}
    \newcommand{\DecValTok}[1]{\textcolor[rgb]{0.25,0.63,0.44}{{#1}}}
    \newcommand{\BaseNTok}[1]{\textcolor[rgb]{0.25,0.63,0.44}{{#1}}}
    \newcommand{\FloatTok}[1]{\textcolor[rgb]{0.25,0.63,0.44}{{#1}}}
    \newcommand{\CharTok}[1]{\textcolor[rgb]{0.25,0.44,0.63}{{#1}}}
    \newcommand{\StringTok}[1]{\textcolor[rgb]{0.25,0.44,0.63}{{#1}}}
    \newcommand{\CommentTok}[1]{\textcolor[rgb]{0.38,0.63,0.69}{\textit{{#1}}}}
    \newcommand{\OtherTok}[1]{\textcolor[rgb]{0.00,0.44,0.13}{{#1}}}
    \newcommand{\AlertTok}[1]{\textcolor[rgb]{1.00,0.00,0.00}{\textbf{{#1}}}}
    \newcommand{\FunctionTok}[1]{\textcolor[rgb]{0.02,0.16,0.49}{{#1}}}
    \newcommand{\RegionMarkerTok}[1]{{#1}}
    \newcommand{\ErrorTok}[1]{\textcolor[rgb]{1.00,0.00,0.00}{\textbf{{#1}}}}
    \newcommand{\NormalTok}[1]{{#1}}
    
    % Additional commands for more recent versions of Pandoc
    \newcommand{\ConstantTok}[1]{\textcolor[rgb]{0.53,0.00,0.00}{{#1}}}
    \newcommand{\SpecialCharTok}[1]{\textcolor[rgb]{0.25,0.44,0.63}{{#1}}}
    \newcommand{\VerbatimStringTok}[1]{\textcolor[rgb]{0.25,0.44,0.63}{{#1}}}
    \newcommand{\SpecialStringTok}[1]{\textcolor[rgb]{0.73,0.40,0.53}{{#1}}}
    \newcommand{\ImportTok}[1]{{#1}}
    \newcommand{\DocumentationTok}[1]{\textcolor[rgb]{0.73,0.13,0.13}{\textit{{#1}}}}
    \newcommand{\AnnotationTok}[1]{\textcolor[rgb]{0.38,0.63,0.69}{\textbf{\textit{{#1}}}}}
    \newcommand{\CommentVarTok}[1]{\textcolor[rgb]{0.38,0.63,0.69}{\textbf{\textit{{#1}}}}}
    \newcommand{\VariableTok}[1]{\textcolor[rgb]{0.10,0.09,0.49}{{#1}}}
    \newcommand{\ControlFlowTok}[1]{\textcolor[rgb]{0.00,0.44,0.13}{\textbf{{#1}}}}
    \newcommand{\OperatorTok}[1]{\textcolor[rgb]{0.40,0.40,0.40}{{#1}}}
    \newcommand{\BuiltInTok}[1]{{#1}}
    \newcommand{\ExtensionTok}[1]{{#1}}
    \newcommand{\PreprocessorTok}[1]{\textcolor[rgb]{0.74,0.48,0.00}{{#1}}}
    \newcommand{\AttributeTok}[1]{\textcolor[rgb]{0.49,0.56,0.16}{{#1}}}
    \newcommand{\InformationTok}[1]{\textcolor[rgb]{0.38,0.63,0.69}{\textbf{\textit{{#1}}}}}
    \newcommand{\WarningTok}[1]{\textcolor[rgb]{0.38,0.63,0.69}{\textbf{\textit{{#1}}}}}
    
    
    % Define a nice break command that doesn't care if a line doesn't already
    % exist.
    \def\br{\hspace*{\fill} \\* }
    % Math Jax compatability definitions
    \def\gt{>}
    \def\lt{<}
    % Document parameters
    \title{eda\_draft}
    
    
    

    % Pygments definitions
    
\makeatletter
\def\PY@reset{\let\PY@it=\relax \let\PY@bf=\relax%
    \let\PY@ul=\relax \let\PY@tc=\relax%
    \let\PY@bc=\relax \let\PY@ff=\relax}
\def\PY@tok#1{\csname PY@tok@#1\endcsname}
\def\PY@toks#1+{\ifx\relax#1\empty\else%
    \PY@tok{#1}\expandafter\PY@toks\fi}
\def\PY@do#1{\PY@bc{\PY@tc{\PY@ul{%
    \PY@it{\PY@bf{\PY@ff{#1}}}}}}}
\def\PY#1#2{\PY@reset\PY@toks#1+\relax+\PY@do{#2}}

\expandafter\def\csname PY@tok@w\endcsname{\def\PY@tc##1{\textcolor[rgb]{0.73,0.73,0.73}{##1}}}
\expandafter\def\csname PY@tok@c\endcsname{\let\PY@it=\textit\def\PY@tc##1{\textcolor[rgb]{0.25,0.50,0.50}{##1}}}
\expandafter\def\csname PY@tok@cp\endcsname{\def\PY@tc##1{\textcolor[rgb]{0.74,0.48,0.00}{##1}}}
\expandafter\def\csname PY@tok@k\endcsname{\let\PY@bf=\textbf\def\PY@tc##1{\textcolor[rgb]{0.00,0.50,0.00}{##1}}}
\expandafter\def\csname PY@tok@kp\endcsname{\def\PY@tc##1{\textcolor[rgb]{0.00,0.50,0.00}{##1}}}
\expandafter\def\csname PY@tok@kt\endcsname{\def\PY@tc##1{\textcolor[rgb]{0.69,0.00,0.25}{##1}}}
\expandafter\def\csname PY@tok@o\endcsname{\def\PY@tc##1{\textcolor[rgb]{0.40,0.40,0.40}{##1}}}
\expandafter\def\csname PY@tok@ow\endcsname{\let\PY@bf=\textbf\def\PY@tc##1{\textcolor[rgb]{0.67,0.13,1.00}{##1}}}
\expandafter\def\csname PY@tok@nb\endcsname{\def\PY@tc##1{\textcolor[rgb]{0.00,0.50,0.00}{##1}}}
\expandafter\def\csname PY@tok@nf\endcsname{\def\PY@tc##1{\textcolor[rgb]{0.00,0.00,1.00}{##1}}}
\expandafter\def\csname PY@tok@nc\endcsname{\let\PY@bf=\textbf\def\PY@tc##1{\textcolor[rgb]{0.00,0.00,1.00}{##1}}}
\expandafter\def\csname PY@tok@nn\endcsname{\let\PY@bf=\textbf\def\PY@tc##1{\textcolor[rgb]{0.00,0.00,1.00}{##1}}}
\expandafter\def\csname PY@tok@ne\endcsname{\let\PY@bf=\textbf\def\PY@tc##1{\textcolor[rgb]{0.82,0.25,0.23}{##1}}}
\expandafter\def\csname PY@tok@nv\endcsname{\def\PY@tc##1{\textcolor[rgb]{0.10,0.09,0.49}{##1}}}
\expandafter\def\csname PY@tok@no\endcsname{\def\PY@tc##1{\textcolor[rgb]{0.53,0.00,0.00}{##1}}}
\expandafter\def\csname PY@tok@nl\endcsname{\def\PY@tc##1{\textcolor[rgb]{0.63,0.63,0.00}{##1}}}
\expandafter\def\csname PY@tok@ni\endcsname{\let\PY@bf=\textbf\def\PY@tc##1{\textcolor[rgb]{0.60,0.60,0.60}{##1}}}
\expandafter\def\csname PY@tok@na\endcsname{\def\PY@tc##1{\textcolor[rgb]{0.49,0.56,0.16}{##1}}}
\expandafter\def\csname PY@tok@nt\endcsname{\let\PY@bf=\textbf\def\PY@tc##1{\textcolor[rgb]{0.00,0.50,0.00}{##1}}}
\expandafter\def\csname PY@tok@nd\endcsname{\def\PY@tc##1{\textcolor[rgb]{0.67,0.13,1.00}{##1}}}
\expandafter\def\csname PY@tok@s\endcsname{\def\PY@tc##1{\textcolor[rgb]{0.73,0.13,0.13}{##1}}}
\expandafter\def\csname PY@tok@sd\endcsname{\let\PY@it=\textit\def\PY@tc##1{\textcolor[rgb]{0.73,0.13,0.13}{##1}}}
\expandafter\def\csname PY@tok@si\endcsname{\let\PY@bf=\textbf\def\PY@tc##1{\textcolor[rgb]{0.73,0.40,0.53}{##1}}}
\expandafter\def\csname PY@tok@se\endcsname{\let\PY@bf=\textbf\def\PY@tc##1{\textcolor[rgb]{0.73,0.40,0.13}{##1}}}
\expandafter\def\csname PY@tok@sr\endcsname{\def\PY@tc##1{\textcolor[rgb]{0.73,0.40,0.53}{##1}}}
\expandafter\def\csname PY@tok@ss\endcsname{\def\PY@tc##1{\textcolor[rgb]{0.10,0.09,0.49}{##1}}}
\expandafter\def\csname PY@tok@sx\endcsname{\def\PY@tc##1{\textcolor[rgb]{0.00,0.50,0.00}{##1}}}
\expandafter\def\csname PY@tok@m\endcsname{\def\PY@tc##1{\textcolor[rgb]{0.40,0.40,0.40}{##1}}}
\expandafter\def\csname PY@tok@gh\endcsname{\let\PY@bf=\textbf\def\PY@tc##1{\textcolor[rgb]{0.00,0.00,0.50}{##1}}}
\expandafter\def\csname PY@tok@gu\endcsname{\let\PY@bf=\textbf\def\PY@tc##1{\textcolor[rgb]{0.50,0.00,0.50}{##1}}}
\expandafter\def\csname PY@tok@gd\endcsname{\def\PY@tc##1{\textcolor[rgb]{0.63,0.00,0.00}{##1}}}
\expandafter\def\csname PY@tok@gi\endcsname{\def\PY@tc##1{\textcolor[rgb]{0.00,0.63,0.00}{##1}}}
\expandafter\def\csname PY@tok@gr\endcsname{\def\PY@tc##1{\textcolor[rgb]{1.00,0.00,0.00}{##1}}}
\expandafter\def\csname PY@tok@ge\endcsname{\let\PY@it=\textit}
\expandafter\def\csname PY@tok@gs\endcsname{\let\PY@bf=\textbf}
\expandafter\def\csname PY@tok@gp\endcsname{\let\PY@bf=\textbf\def\PY@tc##1{\textcolor[rgb]{0.00,0.00,0.50}{##1}}}
\expandafter\def\csname PY@tok@go\endcsname{\def\PY@tc##1{\textcolor[rgb]{0.53,0.53,0.53}{##1}}}
\expandafter\def\csname PY@tok@gt\endcsname{\def\PY@tc##1{\textcolor[rgb]{0.00,0.27,0.87}{##1}}}
\expandafter\def\csname PY@tok@err\endcsname{\def\PY@bc##1{\setlength{\fboxsep}{0pt}\fcolorbox[rgb]{1.00,0.00,0.00}{1,1,1}{\strut ##1}}}
\expandafter\def\csname PY@tok@kc\endcsname{\let\PY@bf=\textbf\def\PY@tc##1{\textcolor[rgb]{0.00,0.50,0.00}{##1}}}
\expandafter\def\csname PY@tok@kd\endcsname{\let\PY@bf=\textbf\def\PY@tc##1{\textcolor[rgb]{0.00,0.50,0.00}{##1}}}
\expandafter\def\csname PY@tok@kn\endcsname{\let\PY@bf=\textbf\def\PY@tc##1{\textcolor[rgb]{0.00,0.50,0.00}{##1}}}
\expandafter\def\csname PY@tok@kr\endcsname{\let\PY@bf=\textbf\def\PY@tc##1{\textcolor[rgb]{0.00,0.50,0.00}{##1}}}
\expandafter\def\csname PY@tok@bp\endcsname{\def\PY@tc##1{\textcolor[rgb]{0.00,0.50,0.00}{##1}}}
\expandafter\def\csname PY@tok@fm\endcsname{\def\PY@tc##1{\textcolor[rgb]{0.00,0.00,1.00}{##1}}}
\expandafter\def\csname PY@tok@vc\endcsname{\def\PY@tc##1{\textcolor[rgb]{0.10,0.09,0.49}{##1}}}
\expandafter\def\csname PY@tok@vg\endcsname{\def\PY@tc##1{\textcolor[rgb]{0.10,0.09,0.49}{##1}}}
\expandafter\def\csname PY@tok@vi\endcsname{\def\PY@tc##1{\textcolor[rgb]{0.10,0.09,0.49}{##1}}}
\expandafter\def\csname PY@tok@vm\endcsname{\def\PY@tc##1{\textcolor[rgb]{0.10,0.09,0.49}{##1}}}
\expandafter\def\csname PY@tok@sa\endcsname{\def\PY@tc##1{\textcolor[rgb]{0.73,0.13,0.13}{##1}}}
\expandafter\def\csname PY@tok@sb\endcsname{\def\PY@tc##1{\textcolor[rgb]{0.73,0.13,0.13}{##1}}}
\expandafter\def\csname PY@tok@sc\endcsname{\def\PY@tc##1{\textcolor[rgb]{0.73,0.13,0.13}{##1}}}
\expandafter\def\csname PY@tok@dl\endcsname{\def\PY@tc##1{\textcolor[rgb]{0.73,0.13,0.13}{##1}}}
\expandafter\def\csname PY@tok@s2\endcsname{\def\PY@tc##1{\textcolor[rgb]{0.73,0.13,0.13}{##1}}}
\expandafter\def\csname PY@tok@sh\endcsname{\def\PY@tc##1{\textcolor[rgb]{0.73,0.13,0.13}{##1}}}
\expandafter\def\csname PY@tok@s1\endcsname{\def\PY@tc##1{\textcolor[rgb]{0.73,0.13,0.13}{##1}}}
\expandafter\def\csname PY@tok@mb\endcsname{\def\PY@tc##1{\textcolor[rgb]{0.40,0.40,0.40}{##1}}}
\expandafter\def\csname PY@tok@mf\endcsname{\def\PY@tc##1{\textcolor[rgb]{0.40,0.40,0.40}{##1}}}
\expandafter\def\csname PY@tok@mh\endcsname{\def\PY@tc##1{\textcolor[rgb]{0.40,0.40,0.40}{##1}}}
\expandafter\def\csname PY@tok@mi\endcsname{\def\PY@tc##1{\textcolor[rgb]{0.40,0.40,0.40}{##1}}}
\expandafter\def\csname PY@tok@il\endcsname{\def\PY@tc##1{\textcolor[rgb]{0.40,0.40,0.40}{##1}}}
\expandafter\def\csname PY@tok@mo\endcsname{\def\PY@tc##1{\textcolor[rgb]{0.40,0.40,0.40}{##1}}}
\expandafter\def\csname PY@tok@ch\endcsname{\let\PY@it=\textit\def\PY@tc##1{\textcolor[rgb]{0.25,0.50,0.50}{##1}}}
\expandafter\def\csname PY@tok@cm\endcsname{\let\PY@it=\textit\def\PY@tc##1{\textcolor[rgb]{0.25,0.50,0.50}{##1}}}
\expandafter\def\csname PY@tok@cpf\endcsname{\let\PY@it=\textit\def\PY@tc##1{\textcolor[rgb]{0.25,0.50,0.50}{##1}}}
\expandafter\def\csname PY@tok@c1\endcsname{\let\PY@it=\textit\def\PY@tc##1{\textcolor[rgb]{0.25,0.50,0.50}{##1}}}
\expandafter\def\csname PY@tok@cs\endcsname{\let\PY@it=\textit\def\PY@tc##1{\textcolor[rgb]{0.25,0.50,0.50}{##1}}}

\def\PYZbs{\char`\\}
\def\PYZus{\char`\_}
\def\PYZob{\char`\{}
\def\PYZcb{\char`\}}
\def\PYZca{\char`\^}
\def\PYZam{\char`\&}
\def\PYZlt{\char`\<}
\def\PYZgt{\char`\>}
\def\PYZsh{\char`\#}
\def\PYZpc{\char`\%}
\def\PYZdl{\char`\$}
\def\PYZhy{\char`\-}
\def\PYZsq{\char`\'}
\def\PYZdq{\char`\"}
\def\PYZti{\char`\~}
% for compatibility with earlier versions
\def\PYZat{@}
\def\PYZlb{[}
\def\PYZrb{]}
\makeatother


    % Exact colors from NB
    \definecolor{incolor}{rgb}{0.0, 0.0, 0.5}
    \definecolor{outcolor}{rgb}{0.545, 0.0, 0.0}



    
    % Prevent overflowing lines due to hard-to-break entities
    \sloppy 
    % Setup hyperref package
    \hypersetup{
      breaklinks=true,  % so long urls are correctly broken across lines
      colorlinks=true,
      urlcolor=urlcolor,
      linkcolor=linkcolor,
      citecolor=citecolor,
      }
    % Slightly bigger margins than the latex defaults
    
    \geometry{verbose,tmargin=1in,bmargin=1in,lmargin=1in,rmargin=1in}
    
    

    \begin{document}
    
    
    \maketitle
    
    

    
    \section{Chevron Data Science Challeng: Predicting
Production}\label{chevron-data-science-challeng-predicting-production}

We are tasked with the challenge to develop a data-driven model for
production as a function of geology and engineering design. We are given
the \textbf{base} spreadsheets, the \textbf{geology} files, and the
\textbf{completions} data along with a \textbf{data dictionary} which
defines the features included.

A general outline of my approach to this multi-variate analysis (MVA)
project will be: 1) Familiarize myself with the data and perform an
exploratory data analysis 2) Take the training dataset and build a
regression algorithm to predict EUR 3) Provide some recommendations for
future work

\subsection{Prelude to the challenge}\label{prelude-to-the-challenge}

"This was an internal data science challenge within Chevron and what
we're looking for is for you to review and provide feedback on how you
would approach this challenge (methodology) and how you would present
your finding to the business (stakeholders). There is not an expectation
that you complete the challenge, but rather as a data scientist how
would you approach this problem."

\section{Exploratory Data Analysis (EDA) for Chevron Wolfcamp MVA
Project}\label{exploratory-data-analysis-eda-for-chevron-wolfcamp-mva-project}

The key of an MVA project is to build a predictive model capable of
interpolating EOR as a function of geological and completions parameters
on a multidimensional space populated with samples from a Wolfcamp
field. Specifically, we would like to know an expected value of EUR,
given a desired set of geological and completions parameter values. In
order to build an efficient model, a first step should always be
familiarizing oneself with the data.

Therefore, in this EDA I would like to interactively explore the data,
in order to see what parameters are available, their ranges,
distributions, proportions of outliers, and whether they are related to
the actual production.

\section{Load libraries and data}\label{load-libraries-and-data}

First, lets start with loading the essential set of libraries and data
itself. Visuals will be mainly done by \textbf{seaborn \& matplotlib},
data will be stored in \textbf{pandas} dataframes.

    \begin{Verbatim}[commandchars=\\\{\}]
{\color{incolor}In [{\color{incolor}28}]:} \PY{k+kn}{import} \PY{n+nn}{pandas} \PY{k}{as} \PY{n+nn}{pd}
         \PY{k+kn}{import} \PY{n+nn}{numpy} \PY{k}{as} \PY{n+nn}{np}
         \PY{k+kn}{import} \PY{n+nn}{seaborn} \PY{k}{as} \PY{n+nn}{sns}
         \PY{k+kn}{import} \PY{n+nn}{pandas\PYZus{}profiling}
         \PY{o}{\PYZpc{}}\PY{k}{matplotlib} inline
         
         \PY{k}{def} \PY{n+nf}{read\PYZus{}data}\PY{p}{(}\PY{n}{part} \PY{o}{=} \PY{l+s+s1}{\PYZsq{}}\PY{l+s+s1}{training}\PY{l+s+s1}{\PYZsq{}}\PY{p}{)}\PY{p}{:}
             \PY{k}{if} \PY{o+ow}{not}\PY{p}{(}\PY{n}{part} \PY{o+ow}{in} \PY{p}{[}\PY{l+s+s1}{\PYZsq{}}\PY{l+s+s1}{training}\PY{l+s+s1}{\PYZsq{}}\PY{p}{,}\PY{l+s+s1}{\PYZsq{}}\PY{l+s+s1}{test}\PY{l+s+s1}{\PYZsq{}}\PY{p}{]}\PY{p}{)}\PY{p}{:}
                 \PY{n+nb}{print}\PY{p}{(}\PY{l+s+s1}{\PYZsq{}}\PY{l+s+s1}{Not a valid part argument}\PY{l+s+s1}{\PYZsq{}}\PY{p}{)}
                 \PY{k}{return} \PY{k+kc}{None}\PY{p}{,}\PY{k+kc}{None}\PY{p}{,}\PY{k+kc}{None}
             \PY{c+c1}{\PYZsh{} Load the csvs below:}
             \PY{n}{geo} \PY{o}{=} \PY{n}{pd}\PY{o}{.}\PY{n}{read\PYZus{}csv}\PY{p}{(}\PY{l+s+s1}{\PYZsq{}}\PY{l+s+s1}{./geology\PYZus{}}\PY{l+s+si}{\PYZpc{}s}\PY{l+s+s1}{.csv}\PY{l+s+s1}{\PYZsq{}} \PY{o}{\PYZpc{}} \PY{n}{part}\PY{p}{)}
             \PY{n}{base} \PY{o}{=} \PY{n}{pd}\PY{o}{.}\PY{n}{read\PYZus{}csv}\PY{p}{(}\PY{l+s+s1}{\PYZsq{}}\PY{l+s+s1}{./base\PYZus{}}\PY{l+s+si}{\PYZpc{}s}\PY{l+s+s1}{.csv}\PY{l+s+s1}{\PYZsq{}} \PY{o}{\PYZpc{}} \PY{n}{part}\PY{p}{)}
             \PY{n}{comp} \PY{o}{=} \PY{n}{pd}\PY{o}{.}\PY{n}{read\PYZus{}csv}\PY{p}{(}\PY{l+s+s1}{\PYZsq{}}\PY{l+s+s1}{./completions\PYZus{}}\PY{l+s+si}{\PYZpc{}s}\PY{l+s+s1}{.csv}\PY{l+s+s1}{\PYZsq{}} \PY{o}{\PYZpc{}} \PY{n}{part}\PY{p}{)}
             
             \PY{k}{return} \PY{n}{geo}\PY{p}{,}\PY{n}{base}\PY{p}{,}\PY{n}{comp}
\end{Verbatim}


    \begin{Verbatim}[commandchars=\\\{\}]
{\color{incolor}In [{\color{incolor}29}]:} \PY{n}{geo\PYZus{}train}\PY{p}{,}\PY{n}{base\PYZus{}train}\PY{p}{,}\PY{n}{comp\PYZus{}train} \PY{o}{=} \PY{n}{read\PYZus{}data}\PY{p}{(}\PY{l+s+s1}{\PYZsq{}}\PY{l+s+s1}{training}\PY{l+s+s1}{\PYZsq{}}\PY{p}{)}
         \PY{n}{geo\PYZus{}test}\PY{p}{,}\PY{n}{base\PYZus{}test}\PY{p}{,}\PY{n}{comp\PYZus{}test} \PY{o}{=} \PY{n}{read\PYZus{}data}\PY{p}{(}\PY{l+s+s1}{\PYZsq{}}\PY{l+s+s1}{test}\PY{l+s+s1}{\PYZsq{}}\PY{p}{)}
         \PY{n}{base\PYZus{}train}\PY{o}{.}\PY{n}{head}\PY{p}{(}\PY{p}{)}
\end{Verbatim}


\begin{Verbatim}[commandchars=\\\{\}]
{\color{outcolor}Out[{\color{outcolor}29}]:}    WellID Subarea                                Operator     County  \textbackslash{}
         0       2       E       PIONEER NATURAL RESOURCES USA INC  GLASSCOCK   
         1       3       D       PIONEER NATURAL RESOURCES USA INC  GLASSCOCK   
         2       5       D  ENDEAVOR ENERGY RESOURCES LIMITED PRTS  GLASSCOCK   
         3       7       F  ENDEAVOR ENERGY RESOURCES LIMITED PRTS  GLASSCOCK   
         4      10       E       PIONEER NATURAL RESOURCES USA INC  GLASSCOCK   
         
            Completion.Date  Completion.Year  Surface.Latitude  Surface.Longitude  \textbackslash{}
         0            37698             2003          31.65753         -101.72135   
         1            37698             2003          31.71566         -101.71787   
         2            38022             2004          31.95401         -101.74911   
         3            37714             2003          31.70024         -101.52357   
         4            37733             2003          31.69134         -101.72307   
         
            Depth.Total.Driller..ft. WB.Spacing.Proxy      {\ldots}          Between\_Zone  \textbackslash{}
         0                      8550              160      {\ldots}       SPBR\_U --> WFMP   
         1                      8550               40      {\ldots}         CLFK --> WFMP   
         2                     10500               20      {\ldots}         CLFK --> WFMP   
         3                      8900               80      {\ldots}         CLFK --> WFMP   
         4                      8575         Over 160      {\ldots}       SPBR\_U --> WFMP   
         
           Frac.Stages Treatment.Records  Fluid.Water..Gals.  Acid..Gals.  \textbackslash{}
         0         3.0                 6                 0.0       3500.0   
         1         3.0                 8                 0.0       5000.0   
         2         1.0                 1            334900.0          0.0   
         3         1.0                 1            247212.0          0.0   
         4         3.0                 8                 0.0       5000.0   
         
            Gel.x.link..Gals.  Other..Gals.  Proppant{\ldots}Total..lbs.  \textbackslash{}
         0           187446.0           0.0                471330.0   
         1           201390.0           0.0                476570.0   
         2                0.0           0.0                553020.0   
         3                0.0           0.0                265000.0   
         4           196636.0           0.0                318180.0   
         
            Fluid{\ldots}Total..Gals.  EUR\_o..Mstb.  
         0              190946.0     40.200000  
         1              206390.0     39.050383  
         2              334900.0     51.856000  
         3              247212.0     22.156000  
         4              201636.0     45.276000  
         
         [5 rows x 22 columns]
\end{Verbatim}
            
    Start by plotting the well locations and organizing by subarea

    \begin{Verbatim}[commandchars=\\\{\}]
{\color{incolor}In [{\color{incolor}30}]:} \PY{n}{sns}\PY{o}{.}\PY{n}{lmplot}\PY{p}{(}\PY{n}{x}\PY{o}{=}\PY{l+s+s2}{\PYZdq{}}\PY{l+s+s2}{Surface.Longitude}\PY{l+s+s2}{\PYZdq{}}\PY{p}{,}\PY{n}{y}\PY{o}{=}\PY{l+s+s2}{\PYZdq{}}\PY{l+s+s2}{Surface.Latitude}\PY{l+s+s2}{\PYZdq{}}\PY{p}{,}\PY{n}{data} \PY{o}{=} \PY{n}{base\PYZus{}train}\PY{p}{,}\PY{n}{fit\PYZus{}reg} \PY{o}{=} \PY{k+kc}{False}\PY{p}{,}\PY{n}{hue} \PY{o}{=} \PY{l+s+s1}{\PYZsq{}}\PY{l+s+s1}{Subarea}\PY{l+s+s1}{\PYZsq{}}\PY{p}{,}\PY{n}{size} \PY{o}{=} \PY{l+m+mi}{8}\PY{p}{)}
\end{Verbatim}


\begin{Verbatim}[commandchars=\\\{\}]
{\color{outcolor}Out[{\color{outcolor}30}]:} <seaborn.axisgrid.FacetGrid at 0x7f7295c75a90>
\end{Verbatim}
            
    \begin{center}
    \adjustimage{max size={0.9\linewidth}{0.9\paperheight}}{output_4_1.png}
    \end{center}
    { \hspace*{\fill} \\}
    
    Look like we have quite distinct spatial clusters described by Subarea
variable. This is important for the crossvalidating the model.
Specifically, if in reality we are interested in the well performances
around a specific area where no wells are drilled, our model should be
able to generalize fairly well. \textbf{Therefore, during
crossvalidation we should make sure folds are constructed such that the
validation wells are within one subarea, and none of training wells
include wells from that area per that fold.} Otherwise we risk leaking
information between the test and train sets. After crossvalidation we
pick the best model and only \emph{then} will train on the whole
\textbf{dataset !}

    \section{Profiling the dataframes}\label{profiling-the-dataframes}

Because we have big dataframes, the exhaustive EDA might take too many
lines of code. However, there is a great package called
\textbf{pandas-profiler} that does this job for you. Therefore we ll
just do the most basic profiling using its powers:

    \begin{Verbatim}[commandchars=\\\{\}]
{\color{incolor}In [{\color{incolor}41}]:} \PY{c+c1}{\PYZsh{} Not going to run cell to suppress output due to missing windows fonts.}
         \PY{c+c1}{\PYZsh{} The output from this cell is *\PYZus{}train\PYZus{}report.html}
         
         \PY{c+c1}{\PYZsh{}import pandas\PYZus{}profiling as pd\PYZus{}pf}
         
         \PY{c+c1}{\PYZsh{}for df,name in zip((base\PYZus{}train,geo\PYZus{}train,comp\PYZus{}train),(\PYZsq{}base\PYZsq{},\PYZsq{}geo\PYZsq{},\PYZsq{}comp\PYZsq{})):}
         \PY{c+c1}{\PYZsh{}    profile = pd\PYZus{}pf.ProfileReport(df,bins=40,check\PYZus{}correlation = False)}
         \PY{c+c1}{\PYZsh{}    profile.to\PYZus{}file(outputfile=\PYZdq{}./\PYZpc{}s\PYZus{}train\PYZus{}report.html\PYZdq{} \PYZpc{} name)}
         \PY{c+c1}{\PYZsh{}    print(name)}
\end{Verbatim}


    \begin{Verbatim}[commandchars=\\\{\}]
{\color{incolor}In [{\color{incolor}32}]:} \PY{n}{base\PYZus{}train}\PY{o}{.}\PY{n}{dtypes} \PY{c+c1}{\PYZsh{} just to show that the object variables will need to be specified using .astype}
\end{Verbatim}


\begin{Verbatim}[commandchars=\\\{\}]
{\color{outcolor}Out[{\color{outcolor}32}]:} WellID                        int64
         Subarea                      object
         Operator                     object
         County                       object
         Completion.Date               int64
         Completion.Year               int64
         Surface.Latitude            float64
         Surface.Longitude           float64
         Depth.Total.Driller..ft.      int64
         WB.Spacing.Proxy             object
         SPBY.Spacing.Proxy           object
         Deepest\_Zone                 object
         Between\_Zone                 object
         Frac.Stages                 float64
         Treatment.Records             int64
         Fluid.Water..Gals.          float64
         Acid..Gals.                 float64
         Gel.x.link..Gals.           float64
         Other..Gals.                float64
         Proppant{\ldots}Total..lbs.      float64
         Fluid{\ldots}Total..Gals.        float64
         EUR\_o..Mstb.                float64
         dtype: object
\end{Verbatim}
            
    Voila! Now the results are stored in htmls within the folder. I cannot
seem to get the statistics or

\subsubsection{Base train:}\label{base-train}

\begin{itemize}
\tightlist
\item
  Impute Acid.Gals with 0 ?
\item
  Might want to transform predictor to a more normal-looking
  distribution. Maybe not if using forests
\item
  Fluid gals is super skewed, needs a transform (log transform or
  box-cox)
\item
  Gel Link needs same as above
\item
  Operator is quite fragmented...Not sure what to do with it. Random
  Forests will take care given we train a lot of trees
\item
  OtherGals is useless, looks like. Low variance variable
\item
  ProppantTotal seems to have strong outliers; otherwise logtransform is
  necessary.
\item
  Subarea use for crossvalidation as a stratifier!
\item
  Where is the promised \textbf{direction} column ? Correlation matrix
  is looking quite interesting..Nothing except fro proppant and total
  fluid is in any correlation; the two are weakly correlated (makes
  sense).
\end{itemize}

\subsubsection{Geo train}\label{geo-train}

\begin{itemize}
\tightlist
\item
  Formation picks are mainly missing; interpolated map intersections
  with wells have weird outliers, need to be taken care of, potentially
  an artefact of interpolatino in Petrel
\item
  MPLW KHW is a constant, throwing it away
\item
  As the MAP and the picks correlate very well, no need to replace MAP
  values with picked, because the fraction of NA is much greated than
  fraction of picks in the picks columns
\end{itemize}

\subsubsection{Comp train}\label{comp-train}

\begin{itemize}
\tightlist
\item
  Missing additive values should potentially be treated as
  'non-specified'
\item
  Depth\_Top or Base should tell where the lateral starts for correction
  of EUR
\item
  Fluid amount skewed
\item
  Fluid units are important for correction! But unnecessary for
  prediction !
\item
  Mesh size has a lot of values missing....We ll try to do a model with
  and without it, using the subset of wells
\item
  Same for pressure breakdown
\item
  Proppant agent amount has too many 0s
\item
  Proppant agent type will most likely be a useless variable (undefined
  category is large \%)
\item
  Remark can be thrown out - useless variable
\item
  Looks like for each well there are multiple treatment records; and
  they are compiled in the base\_train using \textbf{sum} aggregation
\end{itemize}

\subsubsection{TODO}\label{todo}

\begin{itemize}
\tightlist
\item
  Find proxy for well length / pay exposure length. Most likely
  BetweenZone column need to be parsed, and then a difference between
  the picks can be calculated
\item
  Need to merge EUR from base with Comp to look at the correlation
\item
  Need to merge EUR from base with GEO ..
\end{itemize}

    \section{Stage \& treatment records summary
issue}\label{stage-treatment-records-summary-issue}

Exploratory analysis showed that Frac Stages parameter has a weird
exponential distribution, with most common value of Frac Stage == 1.
That might seem suspicious, since generally wells are stimulated in a
multistage fashion. if that's the case, then we need to fix the error in
the \textbf{base\_train} by calculating the number of stages from
\textbf{comp\_train}.

It also looks like the Fluid.Total and Proppant Total, as well as Acid
Gals are not computed correctly in the summary table.

Here's the example for a WellID == 26:

    \begin{Verbatim}[commandchars=\\\{\}]
{\color{incolor}In [{\color{incolor}33}]:} \PY{n}{comp\PYZus{}train}\PY{p}{[}\PY{n}{comp\PYZus{}train}\PY{p}{[}\PY{l+s+s1}{\PYZsq{}}\PY{l+s+s1}{WellID}\PY{l+s+s1}{\PYZsq{}}\PY{p}{]} \PY{o}{==} \PY{l+m+mi}{26}\PY{p}{]}
\end{Verbatim}


\begin{Verbatim}[commandchars=\\\{\}]
{\color{outcolor}Out[{\color{outcolor}33}]:}     WellID  Number  Type  Depth Top  Depth Base  Fluid Amount Fluid Units  \textbackslash{}
         83      26       1  ACID       7972        8185        1500.0         GAL   
         84      26       2  FRAC       7972        8185       63798.0         GAL   
         85      26       3  ACID       7124        7597        2000.0         GAL   
         86      26       4  FRAC       7124        7597       76062.0         GAL   
         87      26       5  ACID       6507        6683        1500.0         GAL   
         88      26       6  FRAC       6507        6683       54642.0         GAL   
         
            Fluid Type  Propping Agent Amount Propping Agent Units Propping Agent Type  \textbackslash{}
         83          A                    0.0            undefined           undefined   
         84  X-LINKGEL               155828.0                   LB                SAND   
         85          A                    0.0            undefined           undefined   
         86  X-LINKGEL               181069.0                   LB                SAND   
         87          A                    0.0            undefined           undefined   
         88  X-LINKGEL               134833.0                   LB                SAND   
         
             Pressure Breakdown  Injection Rate Additive Mesh Size      Remark  
         83                 NaN             NaN      HCL       NaN      7 1/2\%  
         84                 NaN            31.5      NaN     20/40  30\# BORATE  
         85                 NaN             NaN      HCL       NaN      7 1/2\%  
         86                 NaN            31.0      NaN     20/40  20\# BORATE  
         87                 NaN             NaN      HCL       NaN      7 1/2\%  
         88                 NaN            32.0      NaN     20/40  20\# BORATE  
\end{Verbatim}
            
    \begin{Verbatim}[commandchars=\\\{\}]
{\color{incolor}In [{\color{incolor}34}]:} \PY{n}{base\PYZus{}train}\PY{p}{[}\PY{n}{base\PYZus{}train}\PY{p}{[}\PY{l+s+s2}{\PYZdq{}}\PY{l+s+s2}{WellID}\PY{l+s+s2}{\PYZdq{}}\PY{p}{]} \PY{o}{==} \PY{l+m+mi}{26}\PY{p}{]}\PY{o}{.}\PY{n}{drop}\PY{p}{(}\PY{p}{[}\PY{l+s+s1}{\PYZsq{}}\PY{l+s+s1}{Operator}\PY{l+s+s1}{\PYZsq{}}\PY{p}{,}\PY{l+s+s1}{\PYZsq{}}\PY{l+s+s1}{County}\PY{l+s+s1}{\PYZsq{}}\PY{p}{]}\PY{p}{,}\PY{n}{axis}\PY{o}{=}\PY{l+m+mi}{1}\PY{p}{)}
\end{Verbatim}


\begin{Verbatim}[commandchars=\\\{\}]
{\color{outcolor}Out[{\color{outcolor}34}]:}     WellID Subarea  Completion.Date  Completion.Year  Surface.Latitude  \textbackslash{}
         13      26       F            38010             2004           31.7163   
         
             Surface.Longitude  Depth.Total.Driller..ft. WB.Spacing.Proxy  \textbackslash{}
         13         -101.52017                     10550              160   
         
            SPBY.Spacing.Proxy Deepest\_Zone     Between\_Zone  Frac.Stages  \textbackslash{}
         13                 20         WFMP  SPBR\_U --> WFMP          1.0   
         
             Treatment.Records  Fluid.Water..Gals.  Acid..Gals.  Gel.x.link..Gals.  \textbackslash{}
         13                  1                 0.0          0.0           317478.0   
         
             Other..Gals.  Proppant{\ldots}Total..lbs.  Fluid{\ldots}Total..Gals.  EUR\_o..Mstb.  
         13           0.0                384990.0              317478.0     38.495916  
\end{Verbatim}
            
    Let's fix this issue by creating a function that takes \textbf{WellID}
and counting the rows with \textbf{Type} == FRAC, as well as computing
the summaries. We won't replace them, but rather add to the existing
dataframe, and then check the correlation tables. If then a strong
correlation is established between a pair of related features, we will
drop them recursively:

    \begin{Verbatim}[commandchars=\\\{\}]
{\color{incolor}In [{\color{incolor}35}]:} \PY{k}{def} \PY{n+nf}{completions\PYZus{}feats}\PY{p}{(}\PY{n}{df}\PY{p}{,}\PY{n}{WellID} \PY{o}{=} \PY{l+m+mi}{26}\PY{p}{)}\PY{p}{:}
             \PY{n}{n\PYZus{}stages} \PY{o}{=} \PY{n}{df}\PY{p}{[}\PY{n}{df}\PY{p}{[}\PY{l+s+s1}{\PYZsq{}}\PY{l+s+s1}{WellID}\PY{l+s+s1}{\PYZsq{}}\PY{p}{]} \PY{o}{==} \PY{n}{WellID}\PY{p}{]}\PY{p}{[}\PY{l+s+s1}{\PYZsq{}}\PY{l+s+s1}{Type}\PY{l+s+s1}{\PYZsq{}}\PY{p}{]}\PY{o}{.}\PY{n}{isin}\PY{p}{(}\PY{p}{[}\PY{l+s+s1}{\PYZsq{}}\PY{l+s+s1}{FRAC}\PY{l+s+s1}{\PYZsq{}}\PY{p}{,}\PY{l+s+s1}{\PYZsq{}}\PY{l+s+s1}{REFRAC}\PY{l+s+s1}{\PYZsq{}}\PY{p}{]}\PY{p}{)}\PY{o}{.}\PY{n}{sum}\PY{p}{(}\PY{p}{)}
             \PY{n}{n\PYZus{}total\PYZus{}records} \PY{o}{=} \PY{n}{df}\PY{p}{[}\PY{n}{df}\PY{p}{[}\PY{l+s+s1}{\PYZsq{}}\PY{l+s+s1}{WellID}\PY{l+s+s1}{\PYZsq{}}\PY{p}{]} \PY{o}{==} \PY{n}{WellID}\PY{p}{]}\PY{o}{.}\PY{n}{shape}\PY{p}{[}\PY{l+m+mi}{0}\PY{p}{]}
             \PY{n}{total\PYZus{}fluid} \PY{o}{=} \PY{n}{df}\PY{p}{[}\PY{n}{df}\PY{p}{[}\PY{l+s+s1}{\PYZsq{}}\PY{l+s+s1}{WellID}\PY{l+s+s1}{\PYZsq{}}\PY{p}{]} \PY{o}{==} \PY{n}{WellID}\PY{p}{]}\PY{p}{[}\PY{l+s+s1}{\PYZsq{}}\PY{l+s+s1}{Fluid Amount}\PY{l+s+s1}{\PYZsq{}}\PY{p}{]}\PY{o}{.}\PY{n}{sum}\PY{p}{(}\PY{p}{)}
             \PY{n}{total\PYZus{}proppant} \PY{o}{=} \PY{n}{df}\PY{p}{[}\PY{n}{df}\PY{p}{[}\PY{l+s+s1}{\PYZsq{}}\PY{l+s+s1}{WellID}\PY{l+s+s1}{\PYZsq{}}\PY{p}{]} \PY{o}{==} \PY{n}{WellID}\PY{p}{]}\PY{p}{[}\PY{l+s+s1}{\PYZsq{}}\PY{l+s+s1}{Propping Agent Amount}\PY{l+s+s1}{\PYZsq{}}\PY{p}{]}\PY{o}{.}\PY{n}{sum}\PY{p}{(}\PY{p}{)}
             \PY{n}{y} \PY{o}{=} \PY{n}{pd}\PY{o}{.}\PY{n}{DataFrame}\PY{p}{(}\PY{n}{data} \PY{o}{=} \PY{p}{\PYZob{}}\PY{l+s+s1}{\PYZsq{}}\PY{l+s+s1}{NStages}\PY{l+s+s1}{\PYZsq{}}\PY{p}{:}\PY{n}{n\PYZus{}stages}\PY{p}{,}\PY{l+s+s1}{\PYZsq{}}\PY{l+s+s1}{TotalFluid\PYZus{}Comp}\PY{l+s+s1}{\PYZsq{}}\PY{p}{:}\PY{n}{total\PYZus{}fluid}\PY{p}{,}\PY{l+s+s1}{\PYZsq{}}\PY{l+s+s1}{Total\PYZus{}Proppant}\PY{l+s+s1}{\PYZsq{}}\PY{p}{:}\PY{n}{total\PYZus{}proppant}\PY{p}{,}\PY{l+s+s1}{\PYZsq{}}\PY{l+s+s1}{WellID}\PY{l+s+s1}{\PYZsq{}}\PY{p}{:}\PY{n}{WellID}\PY{p}{,}\PY{l+s+s1}{\PYZsq{}}\PY{l+s+s1}{Nrecords}\PY{l+s+s1}{\PYZsq{}}\PY{p}{:}\PY{n}{n\PYZus{}total\PYZus{}records}\PY{p}{\PYZcb{}}\PY{p}{,}\PY{n}{index} \PY{o}{=} \PY{p}{[}\PY{l+m+mi}{1}\PY{p}{]}\PY{p}{)}
             \PY{k}{return}\PY{p}{(}\PY{n}{y}\PY{p}{)}
         \PY{n}{comp\PYZus{}feats} \PY{o}{=} \PY{n}{pd}\PY{o}{.}\PY{n}{concat}\PY{p}{(}\PY{p}{[}\PY{n}{completions\PYZus{}feats}\PY{p}{(}\PY{n}{comp\PYZus{}train}\PY{p}{,}\PY{n}{x}\PY{p}{)} \PY{k}{for} \PY{n}{x} \PY{o+ow}{in} \PY{n}{base\PYZus{}train}\PY{p}{[}\PY{l+s+s1}{\PYZsq{}}\PY{l+s+s1}{WellID}\PY{l+s+s1}{\PYZsq{}}\PY{p}{]}\PY{p}{]}\PY{p}{)}
         \PY{n}{comp\PYZus{}feats}\PY{o}{.}\PY{n}{index} \PY{o}{=} \PY{n+nb}{range}\PY{p}{(}\PY{n}{comp\PYZus{}feats}\PY{o}{.}\PY{n}{shape}\PY{p}{[}\PY{l+m+mi}{0}\PY{p}{]}\PY{p}{)}
         \PY{c+c1}{\PYZsh{}comp\PYZus{}feats = completions\PYZus{}feats(comp\PYZus{}train,26)}
\end{Verbatim}


    \begin{Verbatim}[commandchars=\\\{\}]
{\color{incolor}In [{\color{incolor}36}]:} \PY{n}{comp\PYZus{}feats}\PY{o}{.}\PY{n}{head}\PY{p}{(}\PY{p}{)}
\end{Verbatim}


\begin{Verbatim}[commandchars=\\\{\}]
{\color{outcolor}Out[{\color{outcolor}36}]:}    NStages  TotalFluid\_Comp  Total\_Proppant  WellID  Nrecords
         0        3         190946.0        471330.0       2         6
         1        3         206390.0        476570.0       3         8
         2        3         220670.0        471710.0       5         8
         3        3         126000.0        495000.0       7         4
         4        3         213651.0        472140.0      10         7
\end{Verbatim}
            
    \section{Proxy for length.}\label{proxy-for-length.}

Looks like we are not given a proxy for length of the pay exposure, that
needs to be used to adjust EUR. Therefore, let's see if the DepthTop /
DepthBase from the completions table, or the difference between the zone
picks can give us a proxy.

We need first to figure out whether the picks and perforated base and
depths are given in MD or TVD. Let's build the boxplots for picks, and
see where the depth top / base falls on it, and how long that interval
is.

    \begin{Verbatim}[commandchars=\\\{\}]
{\color{incolor}In [{\color{incolor}37}]:} \PY{c+c1}{\PYZsh{} Look at how the picks are mapped \PYZhy{} depthwise, TVD or MD ?}
         \PY{n}{tmp} \PY{o}{=} \PY{n}{geo\PYZus{}train}\PY{o}{.}\PY{n}{loc}\PY{p}{[}\PY{p}{:}\PY{p}{,}\PY{l+s+s1}{\PYZsq{}}\PY{l+s+s1}{CLEARFORK..MAP.}\PY{l+s+s1}{\PYZsq{}}\PY{p}{:}\PY{l+s+s1}{\PYZsq{}}\PY{l+s+s1}{DEVINOAN\PYZus{}UNC}\PY{l+s+s1}{\PYZsq{}}\PY{p}{]}
         \PY{n}{tmp\PYZus{}melted}  \PY{o}{=} \PY{n}{tmp}\PY{o}{.}\PY{n}{melt}\PY{p}{(}\PY{p}{)}
         \PY{n}{sns}\PY{o}{.}\PY{n}{set}\PY{p}{(}\PY{n}{rc}\PY{o}{=}\PY{p}{\PYZob{}}\PY{l+s+s1}{\PYZsq{}}\PY{l+s+s1}{figure.figsize}\PY{l+s+s1}{\PYZsq{}}\PY{p}{:}\PY{p}{(}\PY{l+m+mi}{9}\PY{p}{,}\PY{l+m+mi}{5}\PY{p}{)}\PY{p}{\PYZcb{}}\PY{p}{)}
         \PY{n}{ax} \PY{o}{=} \PY{n}{sns}\PY{o}{.}\PY{n}{boxplot}\PY{p}{(}\PY{n}{x}\PY{o}{=}\PY{l+s+s1}{\PYZsq{}}\PY{l+s+s1}{variable}\PY{l+s+s1}{\PYZsq{}}\PY{p}{,}\PY{n}{y}\PY{o}{=}\PY{l+s+s1}{\PYZsq{}}\PY{l+s+s1}{value}\PY{l+s+s1}{\PYZsq{}}\PY{p}{,}\PY{n}{data}\PY{o}{=}\PY{n}{tmp\PYZus{}melted}\PY{p}{,}\PY{p}{)}
         \PY{n}{x} \PY{o}{=}\PY{n}{ax}\PY{o}{.}\PY{n}{set\PYZus{}xticklabels}\PY{p}{(}\PY{n}{ax}\PY{o}{.}\PY{n}{get\PYZus{}xticklabels}\PY{p}{(}\PY{p}{)}\PY{p}{,}\PY{n}{rotation}\PY{o}{=}\PY{l+m+mi}{30}\PY{p}{)}
\end{Verbatim}


    \begin{center}
    \adjustimage{max size={0.9\linewidth}{0.9\paperheight}}{output_17_0.png}
    \end{center}
    { \hspace*{\fill} \\}
    
    \begin{Verbatim}[commandchars=\\\{\}]
{\color{incolor}In [{\color{incolor}38}]:} \PY{n}{tmp} \PY{o}{=} \PY{n}{comp\PYZus{}train}\PY{p}{[}\PY{p}{[}\PY{l+s+s1}{\PYZsq{}}\PY{l+s+s1}{Depth Top}\PY{l+s+s1}{\PYZsq{}}\PY{p}{,}\PY{l+s+s1}{\PYZsq{}}\PY{l+s+s1}{Depth Base}\PY{l+s+s1}{\PYZsq{}}\PY{p}{,}\PY{l+s+s1}{\PYZsq{}}\PY{l+s+s1}{WellID}\PY{l+s+s1}{\PYZsq{}}\PY{p}{]}\PY{p}{]}\PY{o}{.}\PY{n}{groupby}\PY{p}{(}\PY{p}{[}\PY{l+s+s1}{\PYZsq{}}\PY{l+s+s1}{WellID}\PY{l+s+s1}{\PYZsq{}}\PY{p}{]}\PY{p}{,}\PY{n}{as\PYZus{}index}\PY{o}{=}\PY{k+kc}{False}\PY{p}{)}\PY{o}{.}\PY{n}{aggregate}\PY{p}{(}\PY{p}{\PYZob{}}\PY{l+s+s1}{\PYZsq{}}\PY{l+s+s1}{Depth Top}\PY{l+s+s1}{\PYZsq{}}\PY{p}{:}\PY{l+s+s1}{\PYZsq{}}\PY{l+s+s1}{min}\PY{l+s+s1}{\PYZsq{}}\PY{p}{,}\PY{l+s+s1}{\PYZsq{}}\PY{l+s+s1}{Depth Base}\PY{l+s+s1}{\PYZsq{}}\PY{p}{:}\PY{l+s+s1}{\PYZsq{}}\PY{l+s+s1}{max}\PY{l+s+s1}{\PYZsq{}}\PY{p}{\PYZcb{}}\PY{p}{)}
         \PY{n}{tmp}\PY{p}{[}\PY{l+s+s1}{\PYZsq{}}\PY{l+s+s1}{Length}\PY{l+s+s1}{\PYZsq{}}\PY{p}{]} \PY{o}{=} \PY{n}{tmp}\PY{p}{[}\PY{l+s+s1}{\PYZsq{}}\PY{l+s+s1}{Depth Base}\PY{l+s+s1}{\PYZsq{}}\PY{p}{]} \PY{o}{\PYZhy{}} \PY{n}{tmp}\PY{p}{[}\PY{l+s+s1}{\PYZsq{}}\PY{l+s+s1}{Depth Top}\PY{l+s+s1}{\PYZsq{}}\PY{p}{]}
         \PY{n}{tmp\PYZus{}melted}  \PY{o}{=} \PY{n}{tmp}\PY{o}{.}\PY{n}{melt}\PY{p}{(}\PY{n}{id\PYZus{}vars}\PY{o}{=}\PY{l+s+s1}{\PYZsq{}}\PY{l+s+s1}{WellID}\PY{l+s+s1}{\PYZsq{}}\PY{p}{)}
         \PY{n}{tmp\PYZus{}melted}\PY{o}{.}\PY{n}{sort\PYZus{}values}\PY{p}{(}\PY{l+s+s1}{\PYZsq{}}\PY{l+s+s1}{WellID}\PY{l+s+s1}{\PYZsq{}}\PY{p}{)}
         \PY{c+c1}{\PYZsh{}tmp}
         \PY{n}{sns}\PY{o}{.}\PY{n}{set}\PY{p}{(}\PY{n}{rc}\PY{o}{=}\PY{p}{\PYZob{}}\PY{l+s+s1}{\PYZsq{}}\PY{l+s+s1}{figure.figsize}\PY{l+s+s1}{\PYZsq{}}\PY{p}{:}\PY{p}{(}\PY{l+m+mi}{9}\PY{p}{,}\PY{l+m+mi}{5}\PY{p}{)}\PY{p}{\PYZcb{}}\PY{p}{)}
         \PY{n}{ax} \PY{o}{=} \PY{n}{sns}\PY{o}{.}\PY{n}{boxplot}\PY{p}{(}\PY{n}{x}\PY{o}{=}\PY{l+s+s1}{\PYZsq{}}\PY{l+s+s1}{variable}\PY{l+s+s1}{\PYZsq{}}\PY{p}{,}\PY{n}{y}\PY{o}{=}\PY{l+s+s1}{\PYZsq{}}\PY{l+s+s1}{value}\PY{l+s+s1}{\PYZsq{}}\PY{p}{,}\PY{n}{data}\PY{o}{=}\PY{n}{tmp\PYZus{}melted}\PY{p}{,}\PY{p}{)}
         \PY{n}{x} \PY{o}{=}\PY{n}{ax}\PY{o}{.}\PY{n}{set\PYZus{}xticklabels}\PY{p}{(}\PY{n}{ax}\PY{o}{.}\PY{n}{get\PYZus{}xticklabels}\PY{p}{(}\PY{p}{)}\PY{p}{,}\PY{n}{rotation}\PY{o}{=}\PY{l+m+mi}{30}\PY{p}{)}
\end{Verbatim}


    \begin{center}
    \adjustimage{max size={0.9\linewidth}{0.9\paperheight}}{output_18_0.png}
    \end{center}
    { \hspace*{\fill} \\}
    
    This looks like a plausible length estimate; we should keep this
dataframe for a merge with the base\_train, and then see whether it
correlates with EUR. \textbf{Need to normalize to completions length}

    \begin{Verbatim}[commandchars=\\\{\}]
{\color{incolor}In [{\color{incolor}44}]:} \PY{n}{base\PYZus{}train\PYZus{}merged} \PY{o}{=} \PY{n}{pd}\PY{o}{.}\PY{n}{merge}\PY{p}{(}\PY{n}{base\PYZus{}train}\PY{p}{,}\PY{n}{tmp}\PY{p}{[}\PY{p}{[}\PY{l+s+s1}{\PYZsq{}}\PY{l+s+s1}{Length}\PY{l+s+s1}{\PYZsq{}}\PY{p}{,}\PY{l+s+s1}{\PYZsq{}}\PY{l+s+s1}{WellID}\PY{l+s+s1}{\PYZsq{}}\PY{p}{]}\PY{p}{]}\PY{p}{,}\PY{n}{on} \PY{o}{=} \PY{l+s+s1}{\PYZsq{}}\PY{l+s+s1}{WellID}\PY{l+s+s1}{\PYZsq{}}\PY{p}{)}
         \PY{n}{base\PYZus{}train\PYZus{}merged} \PY{o}{=} \PY{n}{pd}\PY{o}{.}\PY{n}{merge}\PY{p}{(}\PY{n}{base\PYZus{}train\PYZus{}merged}\PY{p}{,}\PY{n}{comp\PYZus{}feats}\PY{p}{,}\PY{n}{on}\PY{o}{=}\PY{l+s+s1}{\PYZsq{}}\PY{l+s+s1}{WellID}\PY{l+s+s1}{\PYZsq{}}\PY{p}{)}
\end{Verbatim}


    Looks like Frac.Stages and Nstages vary quite a bit... We'll keep them
both, and RandomForest with large numbers of trees will figure which one
is more likely to affect the EUR.

    \begin{Verbatim}[commandchars=\\\{\}]
{\color{incolor}In [{\color{incolor}45}]:} \PY{c+c1}{\PYZsh{} PS. Total proppant should generally increase with time, reflecting the industry trend in Texas. }
         \PY{c+c1}{\PYZsh{} Interestingly enough, Total\PYZus{}Proppant calculated from the comp database does not show that behavior, while the Proppant...Total..lbs. does...}
         \PY{c+c1}{\PYZsh{} Needs to be confirmed...}
         \PY{c+c1}{\PYZsh{} Looks like the numbers were capped before date = 38500}
         
         \PY{n}{base\PYZus{}train\PYZus{}merged} \PY{o}{=}\PY{n}{base\PYZus{}train\PYZus{}merged}\PY{p}{[}\PY{n}{base\PYZus{}train\PYZus{}merged}\PY{p}{[}\PY{l+s+s1}{\PYZsq{}}\PY{l+s+s1}{Proppant...Total..lbs.}\PY{l+s+s1}{\PYZsq{}}\PY{p}{]} \PY{o}{\PYZlt{}} \PY{l+m+mf}{2e6}\PY{p}{]}\PY{o}{.}\PY{n}{sort\PYZus{}values}\PY{p}{(}\PY{n}{by} \PY{o}{=} \PY{l+s+s1}{\PYZsq{}}\PY{l+s+s1}{Completion.Date}\PY{l+s+s1}{\PYZsq{}}\PY{p}{)}
         \PY{n}{base\PYZus{}train\PYZus{}merged}\PY{p}{[}\PY{p}{[}\PY{l+s+s1}{\PYZsq{}}\PY{l+s+s1}{Proppant...Total..lbs.}\PY{l+s+s1}{\PYZsq{}}\PY{p}{,}\PY{l+s+s1}{\PYZsq{}}\PY{l+s+s1}{Total\PYZus{}Proppant}\PY{l+s+s1}{\PYZsq{}}\PY{p}{,}\PY{l+s+s1}{\PYZsq{}}\PY{l+s+s1}{Completion.Date}\PY{l+s+s1}{\PYZsq{}}\PY{p}{]}\PY{p}{]}\PY{o}{.}\PY{n}{plot}\PY{p}{(}\PY{n}{x}\PY{o}{=}\PY{l+s+s1}{\PYZsq{}}\PY{l+s+s1}{Completion.Date}\PY{l+s+s1}{\PYZsq{}}\PY{p}{,}\PY{n}{subplots}\PY{o}{=}\PY{k+kc}{True}\PY{p}{)}
\end{Verbatim}


\begin{Verbatim}[commandchars=\\\{\}]
{\color{outcolor}Out[{\color{outcolor}45}]:} array([<matplotlib.axes.\_subplots.AxesSubplot object at 0x7f7288159e10>,
                <matplotlib.axes.\_subplots.AxesSubplot object at 0x7f729044b7f0>],
               dtype=object)
\end{Verbatim}
            
    \begin{center}
    \adjustimage{max size={0.9\linewidth}{0.9\paperheight}}{output_22_1.png}
    \end{center}
    { \hspace*{\fill} \\}
    
    \begin{Verbatim}[commandchars=\\\{\}]
{\color{incolor}In [{\color{incolor}48}]:} \PY{c+c1}{\PYZsh{} Export the base\PYZus{}merged\PYZus{}train.csv and use this for creating the training algorithm}
         \PY{n}{base\PYZus{}train\PYZus{}merged}\PY{o}{.}\PY{n}{to\PYZus{}csv}\PY{p}{(}\PY{l+s+s1}{\PYZsq{}}\PY{l+s+s1}{base\PYZus{}train\PYZus{}merged.csv}\PY{l+s+s1}{\PYZsq{}}\PY{p}{)}
\end{Verbatim}



    % Add a bibliography block to the postdoc
    
    
    
    \end{document}
